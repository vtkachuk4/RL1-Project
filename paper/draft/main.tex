\documentclass{article}


% if you need to pass options to natbib, use, e.g.:
%     \PassOptionsToPackage{numbers, compress}{natbib}
% before loading neurips_2022


% ready for submission
% \usepackage{neurips_2022}


% to compile a preprint version, e.g., for submission to arXiv, add add the
% [preprint] option:
%     \usepackage[preprint]{neurips_2022}


% to compile a camera-ready version, add the [final] option, e.g.:
    \usepackage[final]{neurips_2022}


% to avoid loading the natbib package, add option nonatbib:
%    \usepackage[nonatbib]{neurips_2022}


\usepackage[utf8]{inputenc} % allow utf-8 input
\usepackage[T1]{fontenc}    % use 8-bit T1 fonts
\usepackage{hyperref}       % hyperlinks
\usepackage{url}            % simple URL typesetting
\usepackage{booktabs}       % professional-quality tables
\usepackage{amsfonts}       % blackboard math symbols
\usepackage{nicefrac}       % compact symbols for 1/2, etc.
\usepackage{microtype}      % microtypography
\usepackage{xcolor}         % colors

%%%%%%%%%%%%%%%%%%%%%%%%%%%%%%%%%%%%%%%%%%%%%%%%%%%%%%%%%%%%%%%%%%%%%%%%%%%%%%%%%%
% things we added

\usepackage{amsmath}
\usepackage{amssymb}
\usepackage{amsthm}
\usepackage{bbm}
\newcommand{\RR}{\mathbb{R}}
\newcommand{\PP}{\mathbb{P}}
\newcommand{\cS}{\mathcal{S}}
\newcommand{\cA}{\mathcal{A}}
\newcommand{\cM}{\mathcal{M}}
\bibliographystyle{abbrvnat}


%%%%%%%%%%%%%%%%%%%%%%%%%%%%%%%%%%%%%%%%%%%%%%%%%%%%%%%%%%%%%%%%%%%%%%%%%%%%%%%%%%

\title{Revisiting the Fuzzy Tilling Activation and How to Set its Hyperparameters}


% The \author macro works with any number of authors. There are two commands
% used to separate the names and addresses of multiple authors: \And and \AND.
%
% Using \And between authors leaves it to LaTeX to determine where to break the
% lines. Using \AND forces a line break at that point. So, if LaTeX puts 3 of 4
% authors names on the first line, and the last on the second line, try using
% \AND instead of \And before the third author name.


\author{%
  Muhammad Gohar Javed \\
  Department of Electrical and Computer Engineering \\
  University of Alberta\\
  % Pittsburgh, PA 15213 \\
  \texttt{javed4@ualberta.ca} \\
  % examples of more authors
  \And
  Tian Xiang Du \\
  Department of Computer Science\\
  University of Alberta\\
  \texttt{tdu@ualberta.ca} \\
  \AND
  Amir Bahmani \\
  Department of Computer Science\\
  University of Alberta\\
  \texttt{bahmani1@ualberta.ca} \\
  \And
  Vlad Tkachuk \\
  Department of Computer Science\\
  University of Alberta\\
  \texttt{vtkachuk@ualberta.ca} \\
  % \And
  % Coauthor \\
  % Affiliation \\
  % Address \\
  % \texttt{email} \\
}


\begin{document}


\maketitle


\begin{abstract}
  Sparsity has been shown to improve model performance on decision making problems with non-stationary data, such as online supervised learning and reinforcement learning (RL).
  Sparsity is when a large number of weights in a neural network are approximately zero.
  The fuzzy tiling activation (FTA) has been proposed to enforce sparsity by design, and has been shown to outperform other activations, such as ReLU and Tanh, that do not enforce sparsity.
  However, a difficulty of using the FTA is that it is sensitive to a new \textit{tiling bound} hyperparameter, which currently requires a search to be set effectively.
  In this work we do two things. 
  First, we reproduce experiments comparing FTA to ReLU using deep Q-learning (DQN) on the LunarLander RL environment, showing that indeed, for this environment FTA is no better than ReLU if the tiling bound is not set appropriately,
  Second, we empirically test if a simple technique of normalizing the activation values passed into the FTA cell can remove the need for setting the tiling bound.


\end{abstract}


\section{Introduction} \label{sec:intro}
Neural networks (NN) have been shown to perform well on several challenging problems \cite{mnih2013playing,silver2017mastering} 
Part of the reason for their success is due to their ability to learn good representations of the input data, which can then be used to effectively generalize.
A representation of an input is defined as the values of all the neurons of the NN when the input is passed into the NN.
The values of all the neurons are commonly called \textit{features}.  
Unfortunately, when the NN is trained online (data recieved sequentially), interference can occur between the representations that are learned CITE.
Interference is loosly defined as when a new learned representation affects the performance of the NN on previously seen data in a negative way (i.e. the old representation becomes worse due to the newly learned one).
A potentially useful intuition for when interference can occur is when the representation for some input $x_1$ is dense (a large number of features are non-zero), 
since then, if another input $x_2$ is sufficiently different (uncorolated) from $x_1$, its representation should likely be quite different.
However, in order for a different representation to be learned for $x_1$ the NN will likely have to update the weights that contribute to the representation of $x_1$, 
thus likely making the representation for $x_1$ worse in order to learn a good representation for $x_2$.

A method that has been found to reduce interference, is to enforce sparse representations (a large number of features are zero) CITE.
Revisiting the example above with inputs $x_1, x_2$, if the representation for $x_1$ is sparse, then the representation that is learned for $x_2$ can use weights that are unused for the representation of $x_1$, 
thus causing no interference between thkkke features for $x_1$ and $x_2$.
The best way to enforce sparsity is an active topic of research 
A discussion of related works can be found in Section \ref{sec:background}.

One way to enforce sparsity is by design, a method which does this is using a \textit{fuzzy tiling activation} (FTA) \cite[]{pan2019fuzzy}.
FTA is an activation function that can be used at each layer of a network, similar to ReLU or Tanh.
Different from ReLU and Tanh is that FTA has a vector output instead of a scalar output.
FTA takes as input a scalar and outputs a one-hot vector.
The output of FTA can be thought of as mapping a scalar $z \in \RR$ to one of $k$ evenly spaced bins in $\RR^k$.
A critical step to using FTA effectively is setting the tiling bound parameter $u$.
This is because $z$ should be in $[-u, u]$ to work well. 
However, just satisfying this is usually not enough for good performance (as one could easily set $u$ very large).
For instance if $u = 10$ and $k = 4$ then the bins would be $[-10, -5), [-5, 0), [0, 5), [5, 10]$ 
and if $z$ is really takes values in $[0,1]$ then it will always map to the same bin (bin $3$ since $[0, 1] \subset [0, 5))$.
Thus, as discussed by \cite[]{pan2019fuzzy}, setting the tiling bound is important; however, no good solution exists for setting it other than performing a search.
The sensitivity to the tiling bound was especially prominant in the work of \cite[]{pan2019fuzzy} when using FTA in deep Q-network (DQN) \cite[]{mnih2013playing} in the LunarLander RL environment.

As such, in this work we aim to do two things:
\begin{enumerate}
  \item We aim to reproduce the FTA vs ReLU using DQN in LunarLander experiments presented in \cite[]{pan2019fuzzy}, to confirm that indeed FTA is sensitive to the tiling bound in this setting.
  \item Next, we aim to test if the simple technique of normalizing the values (using $\tanh$, or batch norm) passed to FTA can remove the need for tuning the tiling bound parameter in FTA.
\end{enumerate}

Experiments related to the first point are shown in section \ref{sub-sec:reproduc experiments}. 

To answer the question in the second point we normalize the values passed into FTA using two methods: passing them into a $\tanh$, using batch norm.
The method of using $tanh$ to normalize the input values was discussed in \cite{pan2019fuzzy}, but not tested.
\cite{pan2019fuzzy} mentioned that using this method might suffer from vanishing gradients, which we hope to confirm from this experiment.
We hypothesis that normalizing the values should solve the need to set tiling bounds using search, since normalizing will ensure that then values passed into FTA are in $[-1, 1]$, thus the tiling bound can always be set to $[-1, 1]$.
Experiments related to the second point are shown in section \ref{sub-sec:normalize experiments}. 


\section{Background} \label{sec:background}
Representation learning can influence learning efficiency, positively through generalization but negatively through interference \cite[]{bullinaria1995, iCaRL2016, le2017, liu2019}.
Learned representations using neural networks are especially vulnerable to interference due to aggressive generalization, as updates to one feature’s estimates change the estimates of later features in unexpected ways, degrading model accuracy. 
Interference in features learned by neural networks is even worse when trained on temporally correlated data \cite[]{liu2020, bengio2020, zhang2022}

Several authors \cite[]{ghiassian2020, liu2019, javed2019, sutton2019;}
have shown that sparse representations can reduce interference in learning feature updates: only a small number of features are active for any given input, 
and thus each update is less likely to interfere with the weights of other features. Sparse feature spaces can still be highly expressive and interpretable, 
as outputs can be traced to the activation of the most input-discriminative features. 
However, learning such sparse representations in the online setting remains an open problem.

Existing strategies to ensure sparsity are to pre-train representations offline using regularizers \cite[]{liu2019} or meta-learning \cite{javed2019}, 
or online methods using regularizers with replay buffers \cite{sutton2019}. While sparsity regularizers do improve learning, 
they often lead to high levels of dead neurons and thus degrades the learned representation over time \cite[]{sutton2019}. 
Kernel representations can also provide sparse representations online, but do not scale well to large problems due to its computational complexity \cite[]{pan2019fuzzy}. 
Therefore, there is motivation to find a simpler method to ensure sparsity in learned representations that is easy to implement, train, and scale to larger input spaces.

FTA is an approach to achieve sparsity by design, 
where an activation function that can be applied to any layer of a neural network provides adjustable sparsity without the need for pre-training or adjustments to the loss function. 
The FTA function aggregates inputs of any feature layer into bins with differentiable, overlapping zones. Much like Tile Coding \cite[]{tilecoding}, 
FTA allows learning to be discriminative between different input values through binning, but unlike Tile Coding, 
FTA maintains input generalization through learning activations within the differentiable regions between bins. 
When applied to DQN and Deep Deterministic Policy Gradient (DDPG) \cite[]{ddpg}, 
FTA has been empirically shown to improve agent performance for both discrete and continuous control environments, 
often outperforming activations like ReLU and Tanh \cite[]{pan2019fuzzy}. Furthermore, FTA-applied networks can often perform better without target networks, 
circumventing the sample inefficiency problem inherent in their use \cite[]{pan2019fuzzy}. 
A recent study by \cite{nnrep} on transferring learned representations to networks performing auxiliary tasks found that FTA-based representations 
transferred better and more consistently than ReLU-based representations for less similar tasks.

\section{Preliminaries} \label{sec:prelims}
In this section we introduce the Fuzzy Tiling Activation (FTA) and formally define the RL setting.

\subsection{Fuzzy Tiling Activation}
First we introduce the simpler \textit{tiling activation} (TA) and then extend it to the FTA.
A TA is a function $\phi: \RR \to \RR^k$, which expects inputs $z \in [-u, u]$ and maps them to one-hot vectors (standard basis) in $\RR^k$.
For example, if $u=10$ and $k=4$, then any $z \in [-10,-5)$ would be mapped to $(1, 0 ,0, 0) \in \RR^k$, any $z \in [-5, 0)$ would be mapped to $(0, 1, 0, 0)$, any $z \in [0, 5)$ would be mapped to $(0, 0, 1, 0)$, and any $z \in [5, 10]$ would be mapped to $(0, 0, 0, 1)$.
\cite{pan2019fuzzy} show that the TA can implemented efficientely as follows.
Assume you want evenly spaced bins of size $\delta > 0$, and $k = 2 u/ \delta$, where WLOG $u$ was chosen such that it is divisable by $\delta$. 
Define the tiling vector 
$$\mathbf{c} = (-u, -u + \delta, l + 2 \delta, \dots, u - 2 \delta, u - \delta) \in \RR^k$$
Then the TA can be defined as
$$\phi(z) = \mathbbm{1} - I_+(\max(\mathbf{c} - z\mathbbm{1}, 0) + \max((z -\delta)\mathbbm{1} - \mathbf{c}, 0))$$
where $\mathbbm{1} \in \RR^k$, and $I_+(\cdot)$ is an indicator function which returns 1 if the input is positive and zero otherwise, and is applied element wise to vectors.

An issue with the TA is that it has zero derivative almost everywhere.
In order to solve this issue \cite{pan2019fuzzy} proposed modifying the TA for a fuzzy tiling activation (FTA).
The FTA can be implemented as follows.
Define a fuzzy version of the indicator function as 
$$I_{\eta, +} = I_+(\eta -x)x + I_+(x - \eta)$$
where $\eta \ge 0$.
Then the FTA can be defined as
$$\phi_\eta(z) = \mathbbm{1} - I_{\eta, +} (\max(\mathbf{c} - z \mathbbm{1}) + \max((z-\delta)\mathbbm{1} - \mathbf{c}, 0)).$$
where $I_{\eta, +}$ is applied element wise to vectors
The reason for the use of fuzzy in the name of FTA can be understood from the definition of $I_{\eta, +}$.
The term $I_+(\eta -x)$ is $1$ when $x < \eta$, while when $x > \eta$, $I_+(x - \eta)$ is $1$.
Thus, when $x < \eta$ we have that $I_{\eta ,+}$ evaluates to $x$, giving smoother (fuzzy) indicator function.
Notice, that this means for $x < \eta$ the derivative is non-zero, allowing us to use backpropagation to train out networks.
Also, when $\eta = 0$ the original indicator function $I_+$ can be recovered.

\subsection{Reinforcement Learning}
The reinforcement learning (RL) setting can be formulated as an markov decision process (MDP).
Formally an MDP is characterized by the tuple $(\cS, \cA, \PP, R, \gamma)$, where $\cS$ is that state space, $cA$ is the action space,
$\PP$ is the transition probability kernel, $R: \cS \times \cA \cS \to \RR$ is the reward function, and $\gamma \in [0, 1]$ is the discount factor.
A policy $\pi: \cS \to \cM_1$ is a map from states to distributions over actions, where $\cM_1$ is the set of all probability measures over actions $\cA$.
Then, an agent (policy) interacts with an MDP as follows.
At each time step $t \in 1, 2 \cdots$ the agent observes a state $s_t \in \cS$.
The agent takes an action according to its policy $a_t \sim \pi(s_t)$.
The agent then transits to its next state according to $s_{t+1} \sim \PP(s, a)$, and recieves a reward $R(s_t, a_t, s_{t+1})$.

An action value function under policy $\pi$ is defined as 
$$Q_\pi(s, a) = \mathbf{E}[G_t|S_t =s, A_t=a; A_{t+1: \infty} \sim \pi]$$
where $G_t = \sum_{t=0}^\infty \gamma^t R(s_t, a_t, s_{t+1})$ is the return.
The goal of the agent is maximize its expected reward from each state in $\cS$.


\section{Experiments} \label{sec:experiments}
Explain the environment we test on. 
How the LunarLander env works (i.e. state, action space, and reward structure)
Explain how DQN works since we use it as our algorithm.

\subsection{Reproducibility Experiment} \label{sub-sec:reproduc experiments}
Explain the network structure and how we change only final layer with FTA and ReLU.
Then explain what hyper-parameter sweeps we did and add the plot showing them (should discuss patterns we saw here for FTA hyper-params and how they confirm those discussed in \cite[]{pan2019fuzzy}).
Show the plot of FTA vs ReLU using the same hyper-params as in \cite[]{pan2019fuzzy} and discuss how we see basically same results as them hopefully.

\subsection{Normalizing Experiment} \label{sub-sec:normalize experiments}
% Explain how we plan to normalize the activation value going into FTA (i.e. $z = \tanh(X w)$).
% I think we can leave this as TODO for final report.
In our experiments, detailed in section \ref{sub-sec:reproduc experiments}, it can be observed that the performance of FTA is quite sensitive to the upper and lower tiling bounds ($u$ and $l=-u$). This reinforces the observation made in \cite{pan2019fuzzy}. In figure \ref{} we can see that the best performance on Lunar Lander was achieved with $u=1$ and worst performance was achieved with $u=100$. It can be inferred intuitively, that for a small value of $u$, many inputs to FTA may be out of its range providing zero gradients. On the other hand, for a large value of $u$, many inputs may activate the same tile. Both resulting in many dead neurons and increasing interference. This means there is a specific value of $u$ which would work best for any specific environment. Since its not straightforward to observe the range of outputs from a Neural Network layer, $u$ can not be set mathematically. It has to be tuned manually by sweeping over a range of values in multiple experiments, dramatically increasing the cost of a project.

We hypothesize that if the inputs to FTA were scaled to a certain range, the performance would be less sensitive to the tiling bounds. We try two different methodologies to achieve this. One, using a Batch Normalization \cite[]{ioffe2015batch} layer before FTA. Two, using a $tanh$ activation layer before FTA. Batch Normalization scales it input $x$ to a learned mean $\beta$ and variance $\gamma$. It is defined by equation \ref{eq:batchnorm}. By using a Batch Norm layer before FTA, we expect it to learn the best $\gamma$ and $\beta$ for any particular tiling bound of FTA.

\begin{equation}
    y = \frac{x-E[x]}{\sqrt{Var[x] + \epsilon}} * \gamma + \beta
    \label{eq:batchnorm}
\end{equation}
 $tanh$, on the other hand, is an activation function which maps the input to a continuous range of values between -1 and 1. The larger the input, the closer the output is to 1 and the smaller the input, the closer the output is to -1. This range of outputs can be controlled by multiplying them with a scalar. It is defined by equation \ref{eq:tanh}. The motivation behind using $tanh$ is to control the range of inputs to FTA, so that a standard tiling bound of $u = -l = 1$ would work in every case.

 \begin{equation}
    y = \frac{e^x-e^{-x}}{e^x+e^{-x}}
    \label{eq:tanh}
 \end{equation}

To test these hypothesis, we run two experiments each with $tanh$ and Batch Norm as the layer preceding FTA on Lunar Lander. One, with the best performing FTA tiling bounds and two, with the worst performing FTA tiling bounds. All other parameters and configurations are kept the same as those in section \ref{sub-sec:reproduc experiments}. Figure \ref{fig:bnvtanh} shows the evaluation learning curves plotting episodic return versus environment time steps for all four of these experiments.

\begin{figure}[h]
    \centering
    \includegraphics{}
    \caption{Evaluation learning curves plotting episodic return versus environment time steps for Batch Norm and $tanh$ as layers preceding FTA.}
    \label{fig:bnvtanh}
\end{figure}


\section{Discussion} \label{sec:discussion}
Discuss how the reproducibility experiment basically matches the results in \cite[]{pan2019fuzzy} paper
Discuss what we find from normalizing activations (i.e. if it worked or not, and if not any hypothesis why might have as to why)


% \section*{References}
\newpage
\bibliography{references}

\end{document}